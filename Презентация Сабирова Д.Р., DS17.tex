% !TEX program = xelatex
\documentclass[12pt,compress]{beamer}
\usetheme{Madrid}
\usefonttheme{professionalfonts}
\setcounter{page}{2} 

\usepackage{pdfpages}
\usepackage{float}
\usepackage{fontspec}
\usepackage{polyglossia}
\setmainlanguage{russian}
\setmainfont{PT Serif} % основной текст (русский)
\setsansfont{Montserrat} % заголовки (латиница и цифры)

\newfontfamily\cyrillicfonttt{Courier New} % Установить шрифт для моноширинного текста

\definecolor{myLilac}{RGB}{214,214,238}
\setbeamercolor{palette tertiary}{fg=myLilac,bg=blue!100!black} %цвета для tertiary-палитры (обычно отвечает за цвет левого верхнего и нижнего блоков навигации на слайде). Текст (fg — foreground) будет белым, фон (bg) будет насыщенно-синим.
\setbeamercolor{background canvas}{bg=white}
\setbeamercolor{block title}{bg=myLilac, fg=black}
\setbeamercolor{alerted text}{fg=red} % Красный текст для alertblock
\setbeamercolor{example text}{fg=green} % Зеленый текст для exampleblock
\setbeamercolor{alertblock title}{bg=red, fg=white} % Заголовок alertblock
\setbeamercolor{exampleblock title}{bg=green, fg=white} % Заголовок exampleblock

\title[Адаптивные подходы]{\sffamily Адаптивные подходы к \\ оптимизации цены подшипниковой продукции}
\subtitle{\sffamily Выпускная работа}
\author{Динара Сабирова}
\date{\normalsize 29 ноября 2025 г.}

\begin{document}
		\begin{frame}[plain]
		\titlepage
	\end{frame}
	
	\AtBeginSection[]
{
	\begin{frame}[plain]
		\frametitle{Оглавление}
		\Large
		\tableofcontents[currentsection]
	\end{frame}
}

	% Слайд с общим оглавлением
	\section{Описание цели и задачи проекта}
	
\begin{frame}[plain]
	\begin{block}{Цель}
		Автоматизированное установление цен  на подшипниковую продукцию.
	\end{block}
	
	\begin{block}{Гипотеза}
		Внедрение модели машинного обучения в процесс ценообразования позволит повысить валовую прибыль по высоколиквидным позициям на 5 - 15\% , за счет ускоренного процесса установки цены и ее дальнейшей оптимизации. 
	\end{block}
	
	\begin{block}{Задача}
		Разработка и внедрение модели для автоматизации процесса ценообразования на высоколиквидные товары в режиме реального времени. 
	\end{block}
\end{frame}
	
\section{AS IS}

\begin{frame}[plain]
	\frametitle{Из чего формируется цена?}
	\begin{figure}
		\includegraphics[width=\linewidth]{Составляющие цены}
	\end{figure}
\end{frame}

\begin{frame}{Формирование цены}
	
	\begin{block}{Модель расчёта цены}
		Цена на подшипниковую продукцию формируется из перечня факторов влияющих на цену:
		\[
		\text{Price} \sim (\text{T}; \text{I}; \text{D}; \text{ER}; \text{C}; \text{M})
		\]
	\end{block}
	
	\begin{itemize}
		\item \textbf{Transport} — расходы на доставку и логистику;
		\item \textbf{Insurance} — страхование поставки;
		\item \textbf{Duties} — таможенные пошлины;
		\item \textbf{Exchange Rates} — валютные колебания и риски;
		\item \textbf{Competitors} — цены конкурентов на аналогичную продукцию;
		\item \textbf{Markup} — фиксированная наценка компании.
	\end{itemize}
	
\end{frame}

\begin{frame}{Детали процесса}
	\begin{itemize}
		\item Источники данных
		\begin{itemize}
			\item Внутренние данные о продажах
			\item Информация о конкурентах 
			\item Рыночные тренды и спрос на продукцию
		\end{itemize}
		\item  Методы анализа
		\begin{itemize}
			\item Статистический анализ продаж
			\item Сравнительный анализ с конкурентами
		\end{itemize}				
		\item Инструменты аналитики
		\begin{itemize}
			\item Excel
		\end{itemize}				
		\item Процессы принятия решений
		\begin{itemize}
			\item Ручное внесение изменений в цене (В среднем 3000 позиций, циклично обновляются в течении 15 дней,  150 позиций в день). 
		\end{itemize}											
	\end{itemize}
\end{frame}

\begin{frame}[plain]
	\frametitle{Участники процесса}
	\begin{figure}
		\includegraphics[width=\linewidth]{Участники процесса}
	\end{figure}
\end{frame}

\section{TO BE}

\begin{frame}[plain]
	\frametitle{Этапы}
	\begin{columns}[c] 
		
		\column{\textwidth}{
			\begin{itemize}
				\item Сбор данных
				\item Предварительная обработка данных
				\item Анализ данных и EDA (Exploratory Data Analysis)
				\item Разработка ML моделей установления цены
				\item Валидация и тестирование моделей
				\item Выбор финального пайплайна
			\end{itemize}
		}
	\end{columns}
\end{frame}

\begin{frame}[plain]
	\frametitle{Основные принципы процесса}
	\begin{figure}
		\includegraphics[width=\linewidth]{tobe}
	\end{figure}
\end{frame}


\section{Методология работы}

\begin{frame}[plain]
	\frametitle{Описание данных}
	\begin{itemize}
		\item Датасет:
		\begin{itemize}
			\item 9 346 строк, 40+ признаков. Источник — исторические продажи подшипниковой продукции и внешние прайс-листы конкурентов за период 03.06.2024 - 30.05.2025.
		\end{itemize}
		\item  Основной таргет:
		\begin{itemize}
			\item Price — фактическая цена реализации за единицу продукции.
		\end{itemize}				
		\item Типы признаков:
		\begin{itemize}
			\item непрерывные: вес, количество в контракте, себестоимость, курс валюты, наценка;
			\item категориальные: тип подшипника, производитель, страна, клиент.
		\end{itemize}				 
		\end{itemize}											
\end{frame}

\begin{frame}[plain]
	\frametitle{EDA: Распределение таргета}
	\begin{figure}
		\includegraphics[width=\linewidth]{Распределение таргета}
	\end{figure}
			\begin{itemize}
		\item Вывод: Большинство цен сконцентрированы в диапазоне до 30 000 ₽, распределение имеет длинный правый хвост, что говорит о наличии выбросов в виде более дорогостоящих позиций. Для наглядности выведен график с логарифмированным таргетом.
	\end{itemize}	
\end{frame}

\begin{frame}[plain]
	\frametitle{EDA: Корреляционная матрица}
	\begin{figure}
		\includegraphics[width=\linewidth]{Корреляционная матрица}
		\end{figure}
	\end{frame}
	
\begin{frame}[plain]
	\frametitle{Вывод}
	\begin{itemize}
		\item Цена по FOB в долларах ↔ Цена\_pricek  = 0.49. Отражает рыночную концентрацию на прибыльных сегментах.
		\item Количество\_pricezpu ↔ Количество\_pricek  = 0.14. Это отражает разную стратегию управления ассортиментом —  ЗПУ делает ставку на стабильные модели, а рынок более динамичен.
		\item Цена\_pricezpu ↔ Количество\_pricek = –0.10. Это отражает конкурентное давление в массовом сегменте — ЗПУ снижает цену, чтобы удерживать позиции в категориях с активной конкуренцией.
		
	\end{itemize}	
\end{frame}

\begin{frame}[plain]
	\frametitle{Бейзлайн (OLS): результаты}
	
	\begin{block}{Уравнение линейной регрессии}
		\[
		\widehat{Y} = \beta_0 + 
		\beta_1 \cdot \text{В} + 
		\beta_2 \cdot \text{С} + 
		\beta_3 \cdot \text{Кол} + 
		\beta_4 \cdot \text{Курс} + \varepsilon
		\]
	\end{block}
	
	\begin{itemize}
		\item R²\textsubscript{train}: 0.84; \quad R²\textsubscript{test}: 0.87
		\item  \textbf{Durbin–Wu–Hausman}: $p < 0.05$ →  не принимаем $H_0$  → переменные эндогенны → нужна IV-регрессия
		\item \textbf{Breusch–Pagan}: $p < 0.05$ → не принимаем  $H_0$ → присутствует гетероскедастичность → нужны робастные ошибки (HC3)
		\item \textbf{Основные значимые факторы}: Цена по FOB в долларах: 0.64, Цена\_pricek: 0.23,   Вес штука\_pricezpu: 0.18,  Количество\_pricezpu: 0.05
	\end{itemize}
	
\end{frame}

\begin{frame}[plain]
	\frametitle{OLS: факт vs предикт}
	\begin{figure}
		\includegraphics[width=\linewidth]{OLS факт vs предикт}
	\end{figure}
\end{frame}

\begin{frame}[plain]
	\frametitle{Модели}
	
	\begin{block}{Описание}
		Для сравнения с бейзлайном были опробованы следующие модели:
		\begin{itemize}
			\item \textbf{Ансамбли}: Bagging, Gradient Boosting;
			\item \textbf{Нейросеть (MLP)};
			\item Цель — улучшить точность прогнозирования относительно OLS.
		\end{itemize}
	\end{block}
	
\end{frame}

\section{Результаты}

\begin{frame}[plain]
	\frametitle{Расширенные модели: OLS, Bagging, Boosting, NN}
	\begin{figure}
		\includegraphics[width=\linewidth]{output(2)}
	\end{figure}
	\begin{itemize}
		\item \textbf{Характеристики целевой переменной:}
	     mean = 1384.73, 25-й перцентиль= 128.00, 50-й перцентиль = 470.00, 75-й перцентиль = 1900.00

	\end{itemize}	
\end{frame}

\begin{frame}[plain]
	\frametitle{Random Forest: факт vs предикт}
	\begin{figure}
		\includegraphics[width=\linewidth]{Random Forest факт vs предикт}
	\end{figure}
\end{frame}

\begin{frame}[plain]
	\frametitle{Вывод}
	\begin{itemize}
		\item На формирование цены влияют Цена по FOB в долларах, Цена\_pricek, Вес штука\_pricezpu, Количество\_pricezpu
		\item Для проведения пилота по набору метрик лучший результат показала модель Random Forest. 
	\end{itemize}	
\end{frame}

\section{Дальнейшее развитие проекта}

\begin{frame}[plain]
		\frametitle{Планы}
	\begin{figure}
		\includegraphics[width=\linewidth]{Итог}
	\end{figure}
\end{frame}

\begin{frame}
	\centering
	{\huge \color{blue} Спасибо за внимание!}
\end{frame}

\end{document}